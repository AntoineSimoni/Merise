\documentclass{beamer}

\usepackage[frenchb]{babel}
\usepackage[T1]{fontenc}
\usepackage[utf8]{inputenc}
\usepackage{graphicx}

\usetheme{Warsaw}


\begin{document}

    \begin{frame}
        Quoi ?
    \newline
    \newline
        Merise est une méthodologie à suivre pour aboutir un projet informatique (généralement orienté base de données).
    \newline
    \newline
        Cette méthode consiste à analyser le cahier de charge initiale,
        afin d'extraire les données à modéliser pour finir par l'élaboration du modèle physique du base de données.
    \end{frame}

    \begin{frame}
        Quand, par Qui, Où ?
        \newline
        \newline
        Création en 1978-79 par Peter Chen, Hubert Tardieu et René Colleti à Aix en Provence
    \end{frame}

    \begin{frame}

        Comment ?
    \newline
    \newline
        Le Modèle Conceptuel de Communication (MCC) :
    \newline
        Modèle de conception insistant sur les relations entre les différents acteurs.
    \newline
    \newline
        Le Modèle Conceptuel de Traitements (MCT) :
    \newline
        Modèle de conception mettant en évidence les choix entre plusieurs options.
    \newline
    \newline
        Le Modèle Conceptuel de Données (MCD) :
    \newline
        Représentation graphique des champs d'une base de données nécessaire à sa conception.

    \end{frame}

\end{document}